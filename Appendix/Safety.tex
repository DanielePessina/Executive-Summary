\section{Safety and Environmental Considerations}
\label{EnvironSection}

\subsection{Laws and Regulations}

\textbf{1. Control of Substances Hazardous to Health (COSHH) Regulations, 2002:}\\
COSHH is the law that requires employers to control substances that are hazardous to the health of an individual. Employees’ exposure to hazardous substances can be significantly reduced by taking the relevant steps, which include i) identifying the health hazards, ii) conducting a risk assessment iii) providing control measures to reduce harm to health iv) ensuring that all control measures are in optimal working order v) planning for emergency situations. Manufacturing companies must provide Material Safety Data Sheets (MSDS), which cover proper use, storage and handling of chemicals in an appropriate manner. This also includes Workplace Exposure Limits (WEL) as a guide for workers to monitor exposure to hazardous substances in the workplace. 
\\
\noindent \textbf{2. Dangerous Substances and Explosive Atmosphere Regulations (DSEAR), 2002:}
Dangerous substances include any substances present in the workplace that have the potential to cause harm to people from the release of an explosion, fire or corrosion of metals. The DSEAR was introduced to ensure the safety of workers from the risks included the hazards formerly stated. 


\subsection{Past Incidents}

\subsubsection{The Toxic White Beaches of Rosignano}
\textbf{Causes:} The Solvay chemical plant in Rosignano has been disposing their by-products (mixture of calcium chloride and limestone) in the sea for decades. 
\\\textbf{Description:} The problem is that mixed in with the calcium chloride and limestone are many toxic chemicals such as mercury, arsenic, cadmium, chromium, lead and ammonia, which are incredibly harmful to humans and animals. This is because limestone, which is one of the raw materials needed for the Solvay process, has impurities consisting of heavy metals. 
\\\textbf{Consequences:} Due to the disposal of toxic chemicals in the beach, the Spiagge Bianche is among the 15 most polluted coastal sites on the Mediterranean Sea according to a report published in 1999 by the United Nations Environment Program. In addition, between 2008 and 2010, the town recorded a mortality rate higher than the regional average for the same period, increasing by 2.2 per cent for men and 8.3 percent for women. Lastly, the frequency of tumors and premature mortality (under age 65) are both above the regional average by several percentage points\citep{whitebeaches}.
\\\textbf{Lessons Learnt:} The purer the limestone used in the Solvay  process the more suitable the limestone is for lessening the impact from the production of soda ash on the environment. Additionally, it is necessary to make sure that if by-products are disposed in the sea, they must  be purified from toxic chemicals in order to not contaminate the waterway. 

\subsubsection{TATA Chemicals Incidents}
\textbf{Causes:} Operation of the plant by means of unsafe work methods. Another incident happened in March 12 2021, when a large fire was ignited in the TATA Chemicals plant which was caused by malfunctioning of electrical equipment within an industrial building.
\\\textbf{Description:}  A contractor employee suffered chemical burns when he was engulfed in hot caustic lime dust. In addition, a worker fell from a walkway 8 feet high and became trapped to his waist  \citep{TATA1}.
\\\textbf{Consequences:} Workers from the TATA chemicals factory got injured due to the lack of safety measures in the plant. TATA chemicals was fined with almost \pounds 350,000  \citep{TATA2}.
\\\textbf{Lessons Learnt:} Both incidents could have been avoided with regular assessment of risks and inspection of work equipment.
\subsubsection{Imperial Sugar Company Dust Explosion and Fire}
\textbf{Causes:} The explosion was fueled by massive accumulations of combustible sugar dust throughout the packaging building. 
\\\textbf{Description:} On February 7, 2008, a huge explosion and fire occurred at the Imperial Sugar refinery northwest of Savannah, Georgia.
\\\textbf{Consequences:} This explosion caused 14 deaths and injuring 38 others, including 14 with serious and life-threatening burns.
\\\textbf{Lessons Learnt:} Combustible dust hazard awareness should be incorporated into employee and member companies’ training programs. Combustible dust characteristics, especially ignition energy and minimum explosible concentration should also be studied. In addition, best practices for minimizing dust accumulation should be incorporate and safe housekeeping practices. Finally, specific combustible dust inspection requirements should be added. 

\subsection{12 Principles of Green Chemistry}\\
\label{section:12 principles}
\textbf{1. Prevention:} It is better to prevent waste than to treat or clean up waste after it has been created. The Solvay process avoids a lot of waste due to the recycle of most of its intermediates such as CO$_2$. Additionally, the implementation of a calcium loop as opposed to ammonia results in the production of a safer byproduct (CaCl$_2$ vs NH$_4$Cl in the modified Solvay process). 

\noindent \textbf{2. Atom economy:} Synthetic methods should be designed to maximize the incorporation of all materials used in the process into the final product. In the synthetic method chosen many of the feeds and byproducts are recycled. 

\noindent \textbf{3. Less hazardous chemical synthesis:} Wherever practicable, synthetic methods should be designed to use and generate substances that possess little or no toxicity to human health and the environment. The synthesis method chosen, avoids the use ammonia and introduces a calcium loop. This was done because ammonia is a much more hazardous chemical which has various negative impacts for the environment and health of humans. By avoiding its use and introducing CaO as a replacement, one of the main hazards of the Solvay process has been prevented.

\noindent \textbf{4. Designing safer chemicals:} Chemical products should be designed to preserve efficacy of function while reducing toxicity. The altered Solvay process route chosen ensured the production of soda ash in the safest method. 

\noindent \textbf{5. Safer solvents and auxiliaries:} The use of auxiliary substances (e.g., solvents, separation agents, etc.) should be made unnecessary wherever possible and, innocuous when used. Solvents used were water-based and therefore non-toxic and non-hazardous. 

\noindent \textbf{6. Design for energy efficiency:} Energy requirements should be recognized for their environmental and economic impacts and should be minimized. Synthetic methods should be conducted at ambient temperature and pressure. Heat integration methods will be implemented where hot streams need to be cooled, preferably using colder streams at ambient temperature. No pressurisation is required for the system; all units will operate at 1 atm. 

\noindent \textbf{7. Use of renewable feedstocks: } A raw material or feedstock should be renewable rather than depleting whenever technically and economically practicable. 

\noindent \textbf{8. Reduce derivatives:} Unnecessary derivatisation should be minimized or avoided if possible, because such steps require additional reagents and can generate waste. The Solvay synthesis step was selected to minimize the steps and intermediate reagents. Extra steps involving the purification of the feedstock was avoided by buying the brine instead of energy-intensive techniques to obtain the desired saline content.

\noindent \textbf{9. Catalysis:} Catalytic reagents (as selective as possible) are superior to stoichiometric reagents. This will be the case for the DSR as steam catalyses the first calcination reaction in R-1. 

\noindent \textbf{10. Design for degradation:} Chemical products should be designed so that at the end of their function they break down into innocuous degradation products and do not persist in the environment. Although Na$_2$CO$_3$ is non-toxic, it is non-biodegradable and is designed in that way intentionally for the consumer (in the production of glass and car segments). 

\noindent \textbf{11. Real-time analysis for pollution prevention:} Analytical methodologies need to be further developed to allow for real-time, in-process monitoring and control prior to the formation of hazardous substances. Control measures such as pressure relief valves, temperature sensors, NO$_X$ concentration sensors and flow control valves will be integrated in Solution A’s next stages of the plant design.

\noindent \textbf{12. Inherently safer chemistry for accident prevention:} Substances and the form of a substance used in a chemical process should be chosen to minimize the potential for chemical accidents, including releases, explosions, and fires. The elimination of NH$_3$ in the modified process is paramount in ensuring inherent safety and so accidents and any hazards involving NH$_3$ will be avoided at all costs. 
\vspace{-20pt}
\subsection{Dow\textquotesingle s F\&EI} \label{dowsappendix}
The Dow\textquotesingle s Fire and Explosion Index is a universal method used to rank the relative fire and explosion risk associated with a chemical process. Process data and material characteristics are used to analyse the hazards of each process unit, whereby penalties and credit factors are determined. Table \ref{tab:dows2} show the F\&EI ranges and the degree of the hazard according to the index range respectively.


\noindent The calculations and basis required to obtain the F\&EI values for each process unit are outlined in Table \ref{tab:dows_calc} below. The chemical with the highest material factor in the process was identified using NFPA ratings (factoring flammability and reactivity) in NFPA 704 or from the Material Safety Data Sheet. These were then used to determine the material factors with the aid of the Material Factor Determination Guide  \citep{lees}. 

\noindent The material factors and the index scores were then determined using the following correlations, Where $MF$ is the material factor and $F_1$ and $F_1$ are the general and special hazard penalties respectively: 

\begin{equation}
    F_1 = 1 + sum \ of\ all\ general\ hazard\ penalties 
\end{equation}

\begin{equation}
        F_2 = 1 + sum \ of \ all \ special \ hazard \ penalties 
\end{equation}
\setlength{\belowdisplayskip}{0pt} \setlength{\belowdisplayshortskip}{0pt}
\setlength{\abovedisplayskip}{0pt} \setlength{\abovedisplayshortskip}{0pt}
    
\begin{equation}
    F_3 = F_1 \times F_2
\end{equation}

\begin{equation}
    F\&EI = MF \times F_3 
\end{equation}
\begin{table}[H]
\centering
\caption{F\&EI table showing the ranges and and corresponding degree of hazard}
\begin{tabular}{|c|c|}
\hline
\textbf{F\&E Index   Range} & \textbf{Degree of Hazard}                             \\ \hline
1-60                        & \cellcolor {\color Light}  \\ \hline
61-96                       & \cellcolor Moderate                      \\ \hline
97-127                      & \cellcolor Intermediate                  \\ \hline
128-158                     & \cellcolor{\color Heavy}  \\ \hline
\textgreater{}159           & \cellcolor{\color Severe} \\ \hline
\end{tabular}
\label{tab:dows2}
\end{table}


\begin{table}[H]
\centering
\caption{\label{tab:dows_calc} Sample calculation for the F\&EI for R-1}
\begin{adjustbox}{width=1.05\textwidth}
\begin{tabular}{|l|l|l|}
\hline
\textbf{General process hazards}                    & \textbf{Penalty}                  & \textbf{Comments}                                                                                                                                                                  \\ \hline
\textbf{Base factor}                                         & \textbf{1}                        & \cellcolor[HTML]{C0C0C0}                                                                                                                                                           \\ \hline
Exothermic chemical reactions                                & 0                                 & \begin{tabular}[c]{@{}l@{}}Reaction is endothermic (thermal \\ decomposition)\end{tabular}                                                                                         \\ \hline
Endothermic processes                                        & 0.4                               & \begin{tabular}[c]{@{}l@{}}Decomposition is endothermic and \\ takes place in a reactor where energy \\ source is provided by combustion of \\ solid e.g. calcination\end{tabular} \\ \hline
Material handling and transfer                               & 0.1                               & \begin{tabular}[c]{@{}l@{}}Feedstock is solid so transportation \\ is not an issue. Low probability \\ of combustibility (decomposes  instead)\end{tabular}                        \\ \hline
Enclosed or indoor process units                             & 0                                 & Plant is outside                                                                                                                                                                   \\ \hline
Access                                                       & 0                                 & \begin{tabular}[c]{@{}l@{}}It is assumed there is good access to \\ the process unit\end{tabular}                                                                                  \\ \hline
Drainage and   spill control                                 & 0                                 & It is assumed there is effective drainage                                                                                                                                          \\ \hline
\textbf{General  process factor}                             & \textbf{1.5}                      & \cellcolor                                                                                                                                                          \\ \hline
\textit{\textbf{Special  process hazards}}                   & \cellcolor \textbf{} & \cellcolor                                                                                                                                                           \\ \hline
\textbf{Base factor}                                         & \textbf{1}                        & \cellcolor                                                                                                                                                         \\ \hline
Toxic materials                                              & 0.4                               & \begin{tabular}[c]{@{}l@{}}Penalty of $0.2 \times N_h$,  where in this \\ case the N$_h$ for CaCO$_3$ is 2.\end{tabular}                                                           \\ \hline
Sub-atmospheric   pressures (\textless{}500 mmHg)            & 0                                 & \begin{tabular}[c]{@{}l@{}}Process unit operates at atmospheric \\ conditions\end{tabular}                                                                                         \\ \hline
\textit{Operation in   or near flammable range}              & \cellcolor         & \cellcolor                                                                                                                                                         \\ \hline
Tank farms   storage flammable liquids                       & 0                                 & The process unit is not for storage                                                                                                                                                \\ \hline
Process upset   or purge failure                             & 0                                 & \begin{tabular}[c]{@{}l@{}}Carbon dioxide is fairly stable until \\ a temperature of 2000°C\end{tabular}                                                                           \\ \hline
Always in   flammable range                                  & 0                                 & \begin{tabular}[c]{@{}l@{}}CaCO\_3 is non-flammable to not \\ applicable\end{tabular}                                                                                              \\ \hline
Dust explosion                                               & 1.25                              & \begin{tabular}[c]{@{}l@{}}There is a   possibility of a dust explosion \\ due to the large amount of CaCO$_3$ required \\ for the process\end{tabular}                            \\ \hline
Pressure                                                     & 0                                 & Operating pressure is at atmospheric pressure                                                                                                                                      \\ \hline
Low   temperature                                            & 0                                 & Temperature of reactor is above 1000°C                                                                                                                                             \\ \hline
\textit{Quantity of   flammable/unstable material}           & \cellcolor         & \cellcolor                                                                                                                                                          \\ \hline
Liquids or   gases in process                                & 0.1                               & Presence of natural   gas in the outer vessel                                                                                                                                      \\ \hline
Combustible   gases in storage                               & 0                                 & No storage in R-1                                                                                                                                                                  \\ \hline
Combustible  solids in storage, dust in process              & 0                                 & No combustible solids   in storage                                                                                                                                                 \\ \hline
Corrosion and   erosion                                      & 0.1                               & \begin{tabular}[c]{@{}l@{}}Corrosion rate of   0.127 mm/year with \\ pitting and local corrosion\end{tabular}                                                                      \\ \hline
Leakage - joints and packing                                 & 0                                 & No leakage from joints and packing assumed                                                                                                                                         \\ \hline
Use of fired equipment                                       & 0.1                               & \begin{tabular}[c]{@{}l@{}}Maximum distance from the leak source \\ and below flash point\end{tabular}                                                                             \\ \hline
Hot oil heat exchange system                                 & 0                                 & No hot oil exchange system                                                                                                                                                         \\ \hline
Rotating equipment                                           & 0                                 & No rotating equipment                                                                                                                                                              \\ \hline
\textbf{Special process hazards factor ($F_2$)}              & \textbf{2.95}                     & \cellcolor                                                                                                                                                           \\ \hline
\textbf{Process unit hazards factor ($F_{1} \times F_{2} = F_{3}$)} & \textbf{4.43}                    & \cellcolor                                                                                                                                                          \\ \hline
\textbf{Fire and explosion index ($F_{3} \times MF = F\&EI$)}       & \textbf{92.93}                                \cellcolor                                         \\ \hline
\end{tabular}
\end{adjustbox}
\end{table}

\subsection{Exposure Limits}
\vspace{-10pt}
\begin{table}[H]
\centering
\caption{Exposure limits for the chemicals in the Solvay process}
\begin{adjustbox}{width=1\textwidth}
\begin{tabular}{|l|l|l|}
\hline
\textbf{Material} & \textbf{Permissible exposure limits}                                                                                                                             & \textbf{Toxicity levels for freshwater fish}                                                                                                                                                               \\ \hline
CaCl$_2$          & \begin{tabular}[c]{@{}l@{}}No specific limits have been established \\ for  CaCl$_2$\end{tabular}                                                                & Lepomis macrochirus: LC50:10650 mg/L/ 96h                                                                                                                                                                  \\ \hline
CaCO$_3$          & \begin{tabular}[c]{@{}l@{}}No specific limits have been established \\ for  CaCO$_3$\end{tabular}                                                                & LC50 \textgreater 56 g/L/ 96h                                                                                                                                                                              \\ \hline
CaO               & \begin{tabular}[c]{@{}l@{}}UK: \\ STEL: 4 mg/m$^3$, 15 min \\ STEL: 6 mg/m$^3$, 15 min \\ TWA: 1 mg/m$^3$, 8 hr \\ TWA: 2 mg/m$^3$, 8 hr\end{tabular}            & \begin{tabular}[c]{@{}l@{}}LC50: = 1070 mg/L,  96h static \\ (Cyprinus carpio)\end{tabular}                                                                                                                \\ \hline
Ca(OH)$_2$        & \begin{tabular}[c]{@{}l@{}}UK: \\ STEL: 4 mg/m$^3$, 15 min \\ STEL: 15 mg/m$^3$, 15 min \\ TWA: 1 mg/m$^3$, 8 hr \\ TWA: 5 mg/m$^3$, 8 hr\end{tabular}           & \begin{tabular}[c]{@{}l@{}}LC50 = 160 mg/L 96h static \\ (Gambusia affinis)\end{tabular}                                                                                                                   \\ \hline
CO$_2$            & \begin{tabular}[c]{@{}l@{}}UK: \\ TWA: 5,000 ppm \\ (9,150 mg/m$^3$) \\ STEL: 15,000 ppm \\ (27,400 mg/m$^3$) \\ TWA: 5,000 ppm \\ (9,000 mg/m$^3$)\end{tabular} &                                                                                                                                                                                                            \\ \hline
Na$_2$CO$_3$      & \begin{tabular}[c]{@{}l@{}}No specific limits have been established for \\ Na$_2$CO$_3$\end{tabular}                                                             & \begin{tabular}[c]{@{}l@{}}Fish: LC50 (96h)               \\ L. macrochius: 300 mg/l \\ Fish: LC50 (96h)  \\ P. promelas : 310-1220 mg/l \\ Crustacea - LC50;         \\ D. magna :  265 mg/L\end{tabular} \\ \hline
NaCl              & \begin{tabular}[c]{@{}l@{}}No specific limits have been established \\ for NaCl\end{tabular}                                                                     & Pimephals prome: LC50: 7650mg/L/ 96h                                                                                                                                                                       \\ \hline
NaHCO$_3$         & \begin{tabular}[c]{@{}l@{}}No specific limits have been established \\ for NaHCO$_3$\end{tabular}                                                                & \begin{tabular}[c]{@{}l@{}}LC50:  8250 - 9000 mg/L, 96hstatic \\ (Lepomis macrochirus)\end{tabular}                                                                                                        \\ \hline
CH$_4$            & \begin{tabular}[c]{@{}l@{}}No specific limits have been established \\ for CH$_4$\end{tabular}                                                                   & LC 50 (Various (Freshwater), 96 h): 27.98 mg/l                                                                                                                                                             \\ \hline
\end{tabular}
\end{adjustbox}
\end{table}

\subsection{NFPA Ratings and Hazard Codes}

%\vspace{-45pt}
\begin{table}[H]
\caption{List of the chemicals their corresponding NFPA ratings and hazard codes}
\centering
\begin{tabular}{|l|l|lll|l|}
\hline
\multirow{\textbf{Material}} & \multirow{\textbf{Use}}                                          & \multicolumn{3}{l|}{\textbf{NFPA Ratings}}                                                              & \multirow{\textbf{Hazard code}} \\ \cline{3-5}
                                   &                                                                        & \multicolumn{1}{l|}{\textbf{Flammability}} & \multicolumn{1}{l|}{\textbf{Health}} & \textbf{Reactivity} &                                       \\ \hline
CaCl$_2$                           & By-product                                                             & \multicolumn{1}{l|}{0}                     & \multicolumn{1}{l|}{1}               & 0                   & H319                                  \\ \hline
CaCO$_3$                           & Reactant                                                               & \multicolumn{1}{l|}{0}                     & \multicolumn{1}{l|}{1}               & 0                   & n/a                                   \\ \hline
CaO                                & Intermediate                                                           & \multicolumn{1}{l|}{0}                     & \multicolumn{1}{l|}{3}               & 1                   & H315, H318, H335                      \\ \hline
Ca(OH)$_2$                         & Intermediate                                                           & \multicolumn{1}{l|}{0}                     & \multicolumn{1}{l|}{3}               & 0                   & H315,H318,H335                        \\ \hline
CO$_2$                             & \begin{tabular}[c]{@{}l@{}}Reactant/\\    \\ Intermediate\end{tabular} & \multicolumn{1}{l|}{0}                     & \multicolumn{1}{l|}{2}               & 0                   & H280                                  \\ \hline
H$_2$O                             & Intermediate                                                           & \multicolumn{1}{l|}{0}                     & \multicolumn{1}{l|}{0}               & 0                   & n/a                                   \\ \hline
Na$_2$CO$_3$                       & Product                                                                & \multicolumn{1}{l|}{0}                     & \multicolumn{1}{l|}{1}               & 0                   & H315, H319                            \\ \hline
NaCl                               & Reactant                                                               & \multicolumn{1}{l|}{0}                     & \multicolumn{1}{l|}{0}               & 0                   & n/a                                   \\ \hline
NaHCO$_3$                          & Intermediate                                                           & \multicolumn{1}{l|}{0}                     & \multicolumn{1}{l|}{1}               & 2                   & n/a                                   \\ \hline
CH$_4$                             & Reactant                                                               & \multicolumn{1}{l|}{4}                     & \multicolumn{1}{l|}{1}               & 0                   & H220 H2280                            \\ \hline
\end{tabular}
\end{table}


\subsection{Hazard Analysis of the Main Process Components}\\

\begin{table}[H]
\centering
\caption{Severity ratings for proposed incidents}
\begin{adjustbox}{width=1.05\textwidth}
\begin{tabular}{|l|l|lll|}
\hline
\multicolumn{1}{|c|}{\multirow{\textbf{S}}} & \multicolumn{1}{c|}{\multirow{\textbf{Severity}}} & \multicolumn{3}{c|}{\textbf{Consequences}}                                                                                                                                                    \\ \cline{3-5} 
\multicolumn{1}{|c|}{}                            & \multicolumn{1}{c|}{}                                   & \multicolumn{1}{c|}{\textbf{To people}} & \multicolumn{1}{c|}{\textbf{To plant/equipment}} & \multicolumn{1}{c|}{\textbf{To environment}}                                                     \\ \hline
I                                                 & Minor                                                   & \multicolumn{1}{l|}{First aid}          & \multicolumn{1}{l|}{Superficial damage}          & No damage                                                                                        \\ \hline
II                                                & Moderate                                                & \multicolumn{1}{l|}{Medical care}       & \multicolumn{1}{l|}{Repair needed}               & Minor annoyance to public                                                                        \\ \hline
III                                               & Serious                                                 & \multicolumn{1}{l|}{Disabling}          & \multicolumn{1}{l|}{Loss of a process item}      & \begin{tabular}[c]{@{}l@{}}Short-term environmental \\ damage\end{tabular}                       \\ \hline
IV                                                & Very Serious                                            & \multicolumn{1}{l|}{One fatality}       & \multicolumn{1}{l|}{Local destruction of plant}  & \begin{tabular}[c]{@{}l@{}}Short-term environmental damage \\ in a significant area\end{tabular} \\ \hline
V                                                 & Severe                                                  & \multicolumn{1}{l|}{Several fatalities} & \multicolumn{1}{l|}{Complete destruction}        & \begin{tabular}[c]{@{}l@{}}Long-term environmental damage \\ in a significant area\end{tabular}  \\ \hline
\end{tabular}
\end{adjustbox}
\end{table}

\begin{table}[H]
\centering
\caption{Likelihood ratings for proposed incidents}
\begin{tabular}{|l|l|l|l|}
\hline
\textbf{L}          & \textbf{Likelihood} & \textbf{Probability}                 & \textbf{Frequency} \\ \hline
\textbf{A} & Improbable          & p$_{once}$ \textless 0.001                & 1 every 1000 years \\ \hline
\textbf{B} & Unlikely            & 0.001 \textless p$_{once}$ \textless 0.01 & 1 every 100 years  \\ \hline
\textbf{C} & Possible            & 0.01 \textless p$_{once}$ \textless 0.1   & 1 every 10 years   \\ \hline
\textbf{D} & Probable            & 0.1 \textless p$_{once}$ \textless 1      & 1 per year         \\ \hline
\textbf{E} & Frequent            & p$_{several}$ = 1                         & Several per year   \\ \hline
\end{tabular}
\end{table}

\begin{figure}[H]
\centering
\includegraphics{Figures/Hazard Table.png}
\caption{Risk assessment matrix}
\end{figure}



\newpage

\includepdf[pages=-,landscape=true]{Figures/Safety hazards.pdf}

\newpage
%\subsection{Waste Streams and Contaminants}

\includepdf[pages=-,landscape=true]{Figures/E.8.pdf}
\label{waste}





