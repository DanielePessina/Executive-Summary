
\section{Business Plan}
\vspace{-10pt}
Solution A Ltd. is looking to disrupt the soda ash industry through the application of 21st century engineering technologies. Considering the implementation of a DSR, carbon capture emerges as a feasible technique to significantly reduce Na$_2$CO$_3$ manufacturing carbon emissions. Additionally, the removal of ammonia as an active process chemical vastly improves process safety and minimises environmental harm from outlet streams. With the effects of climate change ever present and imminent, soda ash manufacture faces an inevitable evolution.


\begin{wrapfigure}{O}{0.5\textwidth}
    \centering
    \includegraphics[width=8cm]{Figures/Soda Ash End-Use by Market.png}\\[0cm]
    \caption{Soda ash end-use by market}
    \label{fig:business1}
    \end{wrapfigure}

\noindent The manufacture of Na$_2$CO$_3$ is a low cost high tonnage process. Proximity to raw materials is a pre-requisite for economic feasibility. Therefore, the first step in the development of the business plan was to identify an optimal plant location. For this, Technique for Order of Preference by Similarity to Ideal Solution (TOPSIS) and Analytic Hierarchy Process (AHP) analyses were employed to determine the ideal country from a selection of viable candidates (Appendix \ref{Location TOPSIS and AHP Analysis}). Domestic demand and competition were the highest weighted factors to consider due to the significant logistic costs associated with large tonnage product transportation. Spain was identified as the strongest candidate, with the specific plant location in the city of Sagunt, Valencia. The northern district of Sagunt is a strategic location on the southern coast, with proximity to a large limestone quarry and brine deposit, a high concentration of glass manufacturers in the region and port access to the North African market: it is ideal from a resource and product distribution perspective. Further, the process$'$ low-carbon objective qualifies the project for funding from the EU Innovation Fund \citep{InnovationFund} and tax credits of 12\% from the Spanish government \citep{SpainTaxcredits}.
\vspace{-10pt}

\subsection{Soda Ash Market Analysis}


\vspace{-5pt}
Soda ash is a staple commodity, with a stable compound annual growth rate (CAGR) of 4\% \citep{ChemIntel} and current global market estimated at \$21 billion \citep{ChemIntel}. Over half of the chemical product is employed in glass production. This is a stimulated industry due to growing demand for glass from end-use sectors including construction and automobile divisions that are meeting the needs of a growing, advanced society. Other key soda ash user industries are seen in Figure \ref{fig:business1}.



 % In market analysis section



\noindent Spanish domestic demand for soda ash is estimated at 1.75 Mt (Appendix \ref{Market sizing}). After accounting for competitors, this indicates that 85\% of plant supply will be met domestically whilst exports to North Africa and other European countries will satisfy the surplus. 

\noindent Europe is a net importer of soda ash\citep{HouseoflordsEU}, with demand steadily increasing. Competitors are likely to become less competitive, considering the EU Emissions Trading System is increasing the taxes on high-carbon processes \citep{EUEmissionsTrading} and tightening environmental restrictions \citep{Zeropollutionactionplan}. Therefore, the pioneering reactor technology and renewed reaction pathway combined with the promising market opportunity presents a prudent business opportunity for Solution A Ltd.
\vspace{-6pt}
\subsection{Plant Economic Feasibility Estimation}
\vspace{-6pt}



A high level economic potential analysis has been conducted on the proposed soda ash process. With the assumption of a 40 year operating lifetime, the preliminary cash flow analysis provides key performance indicators (KPIs) displayed in Table \ref{tab:NPV} in Appendix \ref{Calculated KPIs sect}. The capital expenditure (CAPEX) of the plant located in Spain was approximated at \$750 million based on senior soda ash plants. Though the feasibility analysis is in its preliminary stages, the analysis highlights a lucrative opportunity, with a significant margin in which the process remains profitable. The detailed raw material and product pricing as well as cash flow and costing assumptions can be found in Appendix \ref{sect:business strategy}.

\vspace{-10pt}

