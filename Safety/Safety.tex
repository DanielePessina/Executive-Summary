\section{Safety and Environmental Considerations}   
\vspace{-5pt}
\subsection{Chemical Hazards}
\vspace{-5pt}
The Leblanc process was immediately eliminated from initial considerations due to the formation of toxic intermediaries of sulphates and hydrochloric acid fumes, as well as potential formations of cyanide. Additionally, NH$_3$ was eliminated entirely from the modified Solvay process to reduce explosion and pipe leakage risks. Safety data sheets were gathered for all the components involved in the selected process. The highest material factor and health risk was concentrated amounts of dust, safety precautions will be taken and  measures integrated in relevant plant vicinities. According to the NFPA, almost all active components are stable. Methane used for DSR fuel has the largest material factor owing to its high NFPA flammability score.   
\vspace{-15pt}
\subsection{Process Hazards}
\vspace{-10pt}
A qualitative risk assessment was performed on the main process and original Solvay process. Safety operating conditions were considered including the high temperature DSR vessel at 1000\textdegree C, though this temperature is deemed crucial for the reaction. Hazards are mitigated through introducing an insulating medium surrounding the outer fuel vessel. A key endeavor of the development of novel processes is the alleviation of hazards in and surrounding the plant. Thus, all process units within Solution A\textquotesingle s plant aim to operate at close to atmospheric pressure (with the exception of the carbonator). Operating at higher pressures increases hazard risks from potential chemical leaks, resulting in a higher F\&EI score. The detailed process unit hazards table with reference to the Dow\textquotesingle s Fire and Explosion Index is seen in Appendix \ref{dowsappendix}. It is essential that CaCO$_3$ is stored in cool, well-ventilated conditions, otherwise there is a high risk of dust explosions. It is also vital that concentration sensors for toxic gases (in this case, NO$_X$ and SO$_X$) as well as a pressure relief system are put in place and that the area surrounding the DSR is well-ventilated. These measures help detect any toxic gases and overpressurisation, as well as ensure the safety of surrounding personnel.
\vspace{-15pt}
\subsection{Environmental Impact and Waste Treatment}
\vspace{-10pt}
The 12 principles of green chemistry\citep{12principles} were employed to enable a sustainable process (see Appendix \ref{appendixtopsis}). To meet the 30 kg of CO$_2$ per tonne of soda ash emissions limit, the primary synthesis route was optimised to minimise emissions and CO$_2$ is utilised as an intermediate species in the formation of CaCO$_3$ as part of a recycle loop. High purity CO$_2$ outlet streams are destined for sequestration. By consideration of (UK) laws and regulations and emission limits, waste streams containing solid waste and calcium chloride will be treated accordingly to limit damage to aquatic life. Major waste treatment avenues have been tabulated and can be found in Appendix \ref{waste}.
\newpage